% \RequirePackage{plautopatch} % If use uplatex, enable this line.
\documentclass[twocolumn,paper=a4paper,landscape,fontsize=9pt]{jlreq}
\usepackage{url}
\usepackage{graphicx}

\usepackage[backend=bibtex,sorting=none]{biblatex}
\usepackage{ipsj-tohoku}

\bibliography{cites}

\title{luaLaTeXで情報処理学会東北支部研究会の発表資料をつくる}

\author{Omochice\thanks{1}}

% \TechrepVolNoDate{2021-6}{}{2022/02/21}
% \Copyright{Information Processing Society of Japan}

\begin{document}

\affiliation{1}{情報処理学会東北支部大学}

\maketitle

\begin{abstract}
  これは非公式の情報処理学会東北支部研究会の発表資料テンプレートである。

  uplatex,lualatexで動作することが確認されている。

  なお、このabstractは再定義されているため、プリアンブルに記述すると動作しない。
\end{abstract}

支部ページ\cite{ipsjTohoku}で配布されているファイルがluaLaTeXから読み込めなかったので作成しました。

非公式なものなので使えるかどうかは保証しません。

使い方については、TeXのファイル(main.tex)を見ていただくのが早いかと思われます。

以下に注意点をいくつか挙げます。

\begin{itemize}
  \item 内部でgeometryパッケージを読み込んでいるため、他の方法(jlreqの読み込み時指定など)と鑑賞する可能性があります。
  \item \string\maketiteや\string\thanksを上書きしているため、他のパッケージの読み込みのあとに\string\usepackageするのを推奨します。
  \item \string\columnsepを使用しているため、二段組環境外では動きません。
\end{itemize}


\printbibliography[title=参考文献]

\end{document}
